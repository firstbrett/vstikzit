\documentclass{article}
\usepackage{amsmath}
\usepackage{amssymb}
\usepackage{graphicx}
\usepackage{tikz}
\usepackage[margin=1in]{geometry}

\usetikzlibrary{arrows.meta, positioning, shapes}

\title{Power Transmission \\ Homework 6}
\author{Brett Roberts}
\date{}

\begin{document}

\maketitle

\section*{5.2}
A 200-km, 230-kV, 60-Hz, three-phase line has a positive-sequence series impedance $z = 0.08 + j0.48~\Omega/\text{km}$ and a positive-sequence shunt admittance $y = j3.33 \times 10^{-6}~\text{S/km}$. At full load, the line delivers 250 MW at 0.99 p.f. lagging and at 220 kV. Using the nominal $\pi$ circuit, calculate: (a) the ABCD parameters, (b) the sending-end voltage and current, and (c) the percent voltage regulation.

\begin{enumerate}
    \item[(a)]
    \begin{align*}
        \rightarrow Z &= (z)(l) = (0.08+j0.48)(200) = 16+j96\,\Omega = 97.33 \angle 80.54^\circ \Omega \\
        \rightarrow Y &= (y)(l) = (j3.33 \cdot 10^{-6})(200) = j6.66 \cdot 10^{-4}\,\text{S} = 6.66 \cdot 10^{-4} \angle 90^\circ \text{S} \\
        \rightarrow A &= D = 1 + \frac{YZ}{2} = 1 + \frac{(16+j96)(j6.66 \cdot 10^{-4})}{2} = \boxed{0.9681 \angle 0.317^\circ} \\
        \rightarrow B &= Z = \boxed{16+j96\,\Omega} \\
        \rightarrow C &= Y(1+\frac{YZ}{4}) = (j6.66 \cdot 10^{-4})\left(1 + \frac{(16+j96)(j6.66 \cdot 10^{-4})}{4}\right) = \boxed{6.554 \cdot 10^{-4} \angle 90.16^\circ}
    \end{align*}

    \item[(b)]
    \begin{align*}
        V_{R,\phi} &= \frac{220}{\sqrt{3}} = 127.02 \text{ kV} \qquad S_{3\phi} = \frac{250}{0.99} = 252.53 \text{ MVA} \\
        \rightarrow I_R &= \frac{S_{3\phi}}{\sqrt{3}V_{R,LL}} = \frac{252.53 \cdot 10^6}{\sqrt{3}(220 \cdot 10^3)} \angle (-\cos^{-1}(0.99)) = 662.8 \angle -8.11^\circ \text{A} \\
        \rightarrow V_S &= AV_R + BI_R = (0.9681 \angle 0.317^\circ)(127.02 \cdot 10^3) + (16+j96)(662.8 \angle -8.11^\circ) \\
        &= \boxed{155.3 \angle 23.6 \text{ kV}} \\
        \rightarrow I_S &= CV_R + DI_R = (6.554 \cdot 10^{-4} \angle 90.16^\circ)(127.02 \cdot 10^3) + (0.9681 \angle 0.317^\circ)(662.8 \angle -8.11^\circ) \\
        &= \boxed{635.6 \angle -0.35^\circ \text{ A}}
    \end{align*}

    \item[(c)]
    \begin{align*}
        V_{RFL} &= 127.02 \text{ kV}, \quad V_{RNL} = \frac{|V_S|}{|A|} = \frac{155.3}{0.9681} = 160.42 \text{ kV} \\
        \rightarrow \%VR &= \frac{V_{RNL}-V_{RFL}}{V_{RFL}}(100) = \frac{160.42 - 127.02}{127.02}(100) = \boxed{26.3\%}
    \end{align*}
\end{enumerate}

\section*{5.26}
A 350-km, 500-kV, 60-Hz, three-phase uncompensated line has a positive-sequence series reactance $x = 0.34~\Omega/\text{km}$ and a positive-sequence shunt admittance $y = j4.5 \times 10^{-6}~\text{S/km}$. Neglecting losses, calculate: (a) $Z_c$, (b) $\gamma l$, (c) the ABCD parameters, (d) the wavelength $\lambda$ of the line in kilometers, and (e) the surge impedance loading in MW.

\begin{enumerate}
    \item[(a)] $Z_c = \sqrt{\frac{z}{y}} = \sqrt{\frac{j0.34}{j4.5 \cdot 10^{-6}}} = \boxed{274.87\,\Omega}$
    \item[(b)] $\gamma = \sqrt{zy} = \sqrt{(j0.34)(j4.5 \cdot 10^{-6})} = j1.237 \cdot 10^{-3} \text{ rad/km}$ \\
        $\rightarrow \gamma l = (j1.237 \cdot 10^{-3})(350) = \boxed{j0.4329 \text{ rad}}$
    \item[(c)]
    \begin{align*}
        A=D &= \cosh(\gamma l) = \cos(0.4329) = \boxed{0.9078} \\
        \rightarrow B &= Z_c \sinh(\gamma l) = (274.87) \sinh(j0.4329) = j(274.87)\sin(0.4329) = \boxed{j115.3\,\Omega} \\
        \rightarrow C &= \frac{1}{Z_c}\sinh(\gamma l) = \frac{j}{274.87}\sin(0.4329) = \boxed{j1.526 \cdot 10^{-3}\,\text{S}}
    \end{align*}
    \item[(d)] $\lambda = \frac{2\pi}{\beta} = \frac{2\pi}{1.237 \cdot 10^{-3}} = \boxed{5080 \text{ km}}$
    \item[(e)] SIL $= \frac{V_{LL}^2}{Z_c} = \frac{(500 \cdot 10^3)^2}{274.87} = \boxed{909.5 \text{ MW}}$
\end{enumerate}

\section*{5.27}
Determine the equivalent $\pi$ circuit for the line in Problem 5.26.
\begin{align*}
    Z' &= j115.3\,\Omega \\
    \frac{Y'}{2} &= \frac{j}{Z_c}\tan\left(\frac{\beta l}{2}\right) = \frac{j}{274.87}\tan\left(\frac{0.4329}{2}\right) = \boxed{j7.979 \cdot 10^{-4}\,\text{S}}
\end{align*}

\begin{figure}[h!]
\centering
\begin{tikzpicture}[
    >=Latex, % Arrow tip style
    font=\sffamily,
    component/.style={draw, rectangle, minimum height=1.2cm, minimum width=1.0cm},
    terminal/.style={circle, fill, inner sep=1.5pt}
]
    % Coordinates for input/output and junctions
    \coordinate (in_top) at (-5,2.5);
    \coordinate (in_bot) at (-5,0);
    \coordinate (out_top) at (5,2.5);
    \coordinate (out_bot) at (5,0);
    \coordinate (j1) at (-3,2.5);
    \coordinate (j2) at (3,2.5);
    \coordinate (j3) at (-3,0);
    \coordinate (j4) at (3,0);

    % Main components (nodes)
    \node[component] (Z) at (0,2.5) {$Z'$};
    \node[component] (Y1) at (-3,1.25) {$\frac{Y'}{2}$};
    \node[component] (Y2) at (3,1.25) {$\frac{Y'}{2}$};

    % Component value labels
    \node[above=3mm of Z] {$Z' = j115.3\Omega$};
    \node[below=7mm of Z] {$\frac{Y'}{2} = j7.979 \cdot 10^{-4}\text{S}$};

    % Wires
    \draw (in_top) -- (j1) -- (Z.west);
    \draw (Z.east) -- (j2) -- (out_top);
    \draw (in_bot) -- (out_bot);

    % Shunt connections
    \draw (j1) -- (Y1.north);
    \draw (j3) -- (Y1.south);
    \draw (j2) -- (Y2.north);
    \draw (j4) -- (Y2.south);

    % Input/Output terminals and labels
    \node[terminal] at (in_top) {};
    \node[above=1mm of in_top] {$+$};
    \node[terminal] at (in_bot) {};
    \node[below=1mm of in_bot] {$-$};
    \node at (-5.5, 1.25) {$V_S$};

    \node[terminal] at (out_top) {};
    \node[above=1mm of out_top] {$+$};
    \node[terminal] at (out_bot) {};
    \node[below=1mm of out_bot] {$-$};
    \node at (5.5, 1.25) {$V_R$};

    % Current arrows and labels
    \draw[->] (-4.5, 3) -- (-3.5, 3) node[midway, above] {$I_S$};
    \draw[->] (3.5, 3) -- (4.5, 3) node[midway, above] {$I_R$};

\end{tikzpicture}
\caption{Equivalent $\pi$ circuit for the transmission line in Problem 5.26.}
\label{fig:pi_circuit}
\end{figure}

\end{document}